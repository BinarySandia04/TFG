\documentclass[11pt,a4paper,openright,oneside]{article}
\usepackage{amsfonts, amsmath, amssymb,latexsym,amsthm, mathrsfs, enumerate}
\usepackage[utf8]{inputenc}
\usepackage[catalan]{babel}
\usepackage{csquotes}
\usepackage{epsfig}
\parskip=5pt
\parindent=15pt
\usepackage[margin=1.2in]{geometry}
\usepackage{graphicx}
\usepackage{listings}
\usepackage{tikz}
\selectlanguage{catalan}

\setcounter{page}{0}


\numberwithin{equation}{section}
\newtheorem{teo}{Teorema}[section]
\newtheorem*{teo*}{Teorema}
\newtheorem*{prop*}{Proposici\'o}
\newtheorem*{corol*}{Coro{\l}ari}
\newtheorem{prop}[teo]{Proposici\'o}
\newtheorem{corol}[teo]{Coro{\l}ari}
\newtheorem{lema}[teo]{Lema}
\newtheorem{defi}[teo]{Definici\'o}
\newtheorem{nota}{Notaci\'o}


\theoremstyle{definition}
\newtheorem{prob}[teo]{Problema}
\newtheorem*{sol}{Soluci\'o}
\newtheorem{ex}[teo]{Exemple}
\newtheorem{exs}[teo]{Exemples}
\newtheorem{obs}[teo]{Observaci\'o}
\newtheorem{obss}[teo]{Observacions}

\def\qed{\hfill $\square$}

% \renewcommand{\refname}{Bibliografia}
% --------------------------------------------------
\usepackage{fancyhdr}

\lhead{}
\lfoot{}
\rhead{}
\cfoot{}
\rfoot{\thepage}

\begin{document}

\bibstyle{plain}

\thispagestyle{empty}

\begin{titlepage}
\begin{center}
\begin{figure}[htb]
\begin{center}
\includegraphics[width=6cm]{refname.jpg}
\end{center}
\end{figure}

\vspace*{1cm}
\textbf{\LARGE GRAU DE MATEMÀTIQUES } \\
\vspace*{.5cm}
\textbf{\LARGE Treball final de grau} \\

\vspace*{1.5cm}
\rule{16cm}{0.1mm}\\
\begin{Huge}
\textbf{COMPRESSIÓ DE TENSORS UTILITZANT XARXES D'ANELLS TENSORIALS} \\
\end{Huge}
\rule{16cm}{0.1mm}\\

\vspace{1cm}

\begin{flushright}
\textbf{\LARGE Autor: Aran Roig}

\vspace*{2cm}

\renewcommand{\arraystretch}{1.5}
\begin{tabular}{ll}
\textbf{\Large Director:} & \textbf{\Large Dr. Nahuel Statuto } \\
\textbf{\Large Realitzat a:} & \textbf{\Large  Departament de Matemàtiques   } \\
 & \textbf{\Large (nom del departament)} \\
\\
\textbf{\Large Barcelona,} & \textbf{\Large \today }
\end{tabular}

\end{flushright}

\end{center}
\end{titlepage}


\newpage
\pagenumbering{roman}

% ======== REUM
\section*{Abstract}

Goldbach's weak conjecture asserts that every odd integer greater than 5 is the sum of three primes. We study that problem and the proof of it presented by H. A. Helfgott and D. Platt. We focus on the circle method. Finally, we describe a computation that confirms Goldbach's weak conjecture up to $10^{28}$.

\section*{Resum}
La conjectura feble de Goldbach afirma que tot nombre enter imparell major que 5 \'es la suma de tres nombres primers. En aquest treball estudiem aquest problema i la seva prova presentada per HA Helfgott i D. Platt. Ens centrem en el m\`etode del cercle. Finalment, describim un c\`alcul que confirma la conjectura feble de Goldbach fins a $10^{28}$.
{\let\thefootnote\relax\footnote{2020 Mathematics Subject Classification. 11G05, 11G10, 14G10}}


\newpage

% ========= AGRAIMENTS
\section*{Agra\"{\i}ments}

Vull agrair a ... algu algu
\newpage

% ========= CONTENTS
\tableofcontents

\newpage

\pagenumbering{arabic}
\setcounter{page}{1}

% =========== INTRODUCCIÓ
\section*{Introducci\'o}
\iffalse
\begin{prop}
\end{prop}
\begin{teo}
\end{teo}
\begin{prob}
\end{prob}
\begin{ex}
\end{ex}
\begin{defi}
\end{defi}
\begin{lema}
\end{lema}
\fi

\subsection*{El projecte}

La conjectura forta de Goldbach assegura que tot nombre enter parell m\' es gran que 2 pot ser escrit com a suma de dos nombres primers. \'es f\`acil veure que aquesta conjectura implica l'anomenada conjectura feble de Goldbach: tot nombre enter senar m\'es gran que 5 pot ser escrit com a suma d'exactament tres nombres primers.

En aquest treball comentarem l'origen d'aquests problemes, exposarem una cro\-no\-lo\-gia parcial del seu tractament, amb alguns dels resultats m\'es coneguts, i n'enunciarem algunes conseq\"u\`encies.

Posarem de manifest les idees b\`asiques i les t\`ecniques utilitzades en l'estudi de la conjectura feble de Goldbach, centrant-nos en les de la demostraci\'o presentada recentment per H. A. Helfgott en co{\l}aboraci\'o amb D. Platt, prestant especial atenci\'o al m\`etode del cercle de Hardy-Littlewood.

Una part for\c{c}a extensa de la tasca realitzada consisteix en una verificaci\'o parcial de la conjectura feble de Goldbach fins a $10^{28}$, que requereix del c\`alcul d'uns dos mil cinc-cents milions de nombres primers certificats. Veurem com portar a la pr\`actica tals c\`alculs i guardar les dades generades utilitzant els recursos inform\`atics de qu\`e disposem.


\subsection*{Estructura de la Mem\`oria}

\newpage

% ================ PRELIMINARS

\section{Xarxes Tensorials}



\begin{defi}
 Un \textbf{tensor} de ordre $d$ es una matriu multidimensional $T \in \mathbb{R}^{N_1 \times \dots \times N_d}$ 
Representarem l'element indexat per $\{i_1, \dots, i_n\}$ de $T$ com $T_{i_1 \dots i_n}$
\end{defi}

\noindent Sigui $W$ una matriu sobre $\mathbb{R}$ de dimensions $M \times N$, amb $M = \prod_{i=1}^d M_i$, $N = \prod_{i=1}^p N_i$. Podem
remodelar aquesta matriu com a un tensor $T \in \mathbb{R}^{M_1 \times \dots \times M_d \times N_1 \times \dots \times N_p}$ amb la funció bijectiva


Al llarg d'aquest treball representarem els tensors i les xarxes tensorials a partir de representacions diagramiques. Aquestes
representacions ens seràn més útils per estalviar notació a l'hora de tractar contraccions tensorials

\begin{figure}[h]
\begin{center}
    $$A = \begin{bmatrix}
        a_1 \\
        a_2 \\
        \vdots \\
        a_n
    \end{bmatrix} \qquad
    B = \begin{bmatrix}
        b_{11} & \dots & b_{1m} \\
        \vdots & & \vdots \\
        b_{n1} & \dots & b_{nm}
    \end{bmatrix}
    \qquad
    C=\begin{bmatrix}
        c_{111} & \dots & c_{1m1} \\
        \vdots & & \vdots \\
        c_{n11} & \dots & c_{nm1} 
    \end{bmatrix}
    \dots
\begin{bmatrix}
        c_{11k} & \dots & c_{1mk}\\
        \vdots & & \vdots \\
        c_{n1k} & \dots & c_{nmk} 
    \end{bmatrix}
    \
    $$

\tikz{
    \node at (-1,-0.5) {$A_i \Leftrightarrow$};
    \node[draw, shape=circle] (v0) at (0,0) {$A$};
    \node(i) at (0,-1) {$i$};
    \draw (v0) -- (i);
}\qquad \tikz{
    \node at (-1,-0.5) {$B_{ij} \Leftrightarrow$};
    \node[draw, shape=circle] (v0) at (0,0) {$B$};
    \node(i) at (-0.5,-1) {$i$};
    \node(j) at (0.5,-1) {$j$};
    \draw (v0) -- (i);
    \draw (v0) -- (j);
}\qquad \tikz{
    \node at (-1,-0.5) {$C_{ijk} \Leftrightarrow$};
    \node[draw, shape=circle] (v0) at (0,0) {$C$};
    \node(i) at (-0.5,-1) {$i$};
    \node(j) at (0,-1) {$j$};
    \node(k) at (0.5,-1) {$k$};
    \draw (v0) -- (i);
    \draw (v0) -- (j);
    \draw (v0) -- (k);
}
\caption{
    Representació diagramàtiques de tensors $A,B,C$ d'ordres $1,2$ i $3$ respectivament. Els nodes representen tensors i cada aresta desconectada
    representa un index.
}
\end{center}
\label{fig:tnot}
\end{figure}

\begin{defi}
    Donats dos tensors $T \in \mathbb{R}^{n_1 \times \dots \times n_d}$ i $P \in \mathbb{R}^{m_1 \times \dots \times m_p}$ d'ordres $d$ i $p$
    respectivament tals que $n_a = m_b$, definim la \textbf{contracció} de $T$ i $P$ respecte els indexos $a$ i $b$ com al tensor $Q$ de dimensió
    $\prod_{i \neq a}^{d} n_i \times \prod_{j \neq b}^{p} m_j$ i ordre $d + p - 2$ indexat de la forma
    $$ (Q_{(i_k)_{k \neq a},(j_k)_{k \neq b}}) = \sum_{i_a=j_b=1}^{n_a} T_{i_1\cdots i_a \cdots i_d} P_{j_1 \cdots j_b \cdots j_p}$$
\end{defi}

\begin{figure}[h]
    \begin{center}
        \begin{tikzpicture}
            \node(T)[draw, shape=circle] at (0, 0) {T};
            \node(P)[draw, shape=circle] at (2, 0) {P};
            \draw (T) -- node[midway, above]{n} (P);
            \draw (T) -- node[right]{$I_2$} (0, -1);
            \draw (P) -- node[right]{$I_3$} (2, -1);
            \draw (T) -- node[above]{$I_1$} (-1, 0);
            \draw (3,0) -- node[above]{$I_4$} (P);
        \end{tikzpicture}
        \caption{
            Representació diagramàtica d'una xarxa tensorial. Cada aresta connectada entre dos tensors representa una contracció o un
            sumatori respecte d'un índex. Per conveni, els indexos s'ordenen en sentit horari començant des de baix. En aquest cas, el tensor resultant de la contracció s'indexaria de la forma 
            $(Q_{i_1i_2i_3i_4}) = \sum_{j=1}^{n} T_{i_2i_1j} P_{i_3ji_4}$.
        }
    \end{center}
    \label{fig:tcont}
\end{figure}

La nostra motivació consistirà en descomposar i representar tensors de dimensionalitat (o d'ordre) alt per seqüencies de contraccions de tensors de més baixa dimensionalitat

TODO: Que es un espai latent? Latent dimensions?
Explicar normes tensorials



\newpage

\section{Conclusions}

Fent servir un s\'{\i}mil geom\`etrico-cartogr\`afic, aquesta mem\`oria constitueix un mapa a escala planet\`aria de la demostraci\'o de la conjectura feble de Goldbach presentada per Helfgott i un mapa a escala continental de la verificaci\'o num\`erica d'aquesta. Estudis posteriors i m\'es profunds haurien de permetre elaborar mapes de menor escala.

La naturalesa dels nombres primers ens ha portat per molts racons diferents de les Matem\`atiques; en no imposar-nos restriccions en la forma de pensar, hem pogut gaudir del viatge i assolir els objectius que ens vam plantejar a l'inici del projecte i anar m\'es enll\`a, sobretot en el camp de la computaci\'o i la manipulaci\'o de grans volums de dades num\`eriques.

Una gran part dels coneixements b\`asics que hem hagut de fer servir han estat treballats en les assignatures de M\`etodes anal\'{\i}tics en teoria de nombres i d'An\`alisi harm\`onica i teoria del senyal, que s\'on optatives de quart curs del Grau de Ma\-te\-m\`a\-ti\-ques. Altres els hem hagut d'aprendre durant el desenvolupant del projecte. S'ha realitzat una tasca de recerca bibliogr\`afica important, consultant recursos antics i moderns, tant en format digital com en format paper.

\normalfont

\newpage

\begin{thebibliography}{25}
\bibitem{wide} Wenqi Wang; Yifan Sun; Brian Eriksson; Wenlin Wang; Vaneet Aggarwal: Wide Compression: Tensor Ring Nets, \texttt{arXiv:1802.09052v1}, 2018 
\bibitem{trd} Qibin Zhao, Guoxu Zhou, Shengli Xie, Liqing Zhang, Andrzej Cichocki: Tensor Ring Decomposition \texttt{arXiv:1606.05535v1 [cs.NA]}, 17 Jun 2016
\bibitem{guide} Glen Evenbly: A Practical Guide to the Numerical Implementation of Tensor Networks I: Contractions, Decompositions and Gauge Freedom \texttt{arXiv:2202.02138v1 [quant-ph]}, 4 Feb 2022
%\bibitem{pari} Batut, C.; Belabas, K.; Bernardi, D.; Cohen, H.; Olivier, M.: User's guide to \textit{PARI-GP},  \newline \texttt{pari.math.u-bordeaux.fr/pub/pari/manuals/2.3.3/users.pdf}, 2000.
%\bibitem{cw} Chen, J. R.; Wang, T. Z.: On the Goldbach problem, \textit{Acta Math. Sinica}, 32(5):702-718, 1989.
%\bibitem{desh} Deshouillers, J. M.: Sur la constante de $\check{\text{S}}\text{nirel}^{\prime} \text{man}$, \textit{S\'eminaire Delange-Pisot-Poitou, 17e ann\'ee: (1975/76), Th\'eorie des nombres: Fac. 2, Exp. No.} G16, p\`ag. 6, Secr\`etariat Math., Paris, 1977.
%\bibitem{derz} Deshouillers, J. M.; Effinger, G.; te Riele, H.; Zinoviev, D.: A complete Vinogradov 3-primes theorem under the Riemann hypothesis, \textit{Electron. Res. Announc. Amer. Math. Soc.}, 3:99-104, 1997.
%\bibitem{dick} Dickson, L. E.: \textit{History of the theory of numbers. Vol. I: Divisibility and primality}, Chelsea Publishing Co., New York, 1966.
%\bibitem{hl} Hardy, G. H.; Littlewood, J. E.: Some problems of \textquoteleft Partitio numerorum\textquoteright; III: On the expression of a number as a sum of primes, \textit{Acta Math.}, 44(1):1-70, 1923.
%\bibitem{hara} Hardy, G. H.; Ramanujan, S.: Asymptotic formulae in combinatory analysis, \textit{Proc. Lond. Math. Soc.}, 17:75-115, 1918.
%\bibitem{haw} Hardy, G. H.; Wright, E. M.: \textit{An introduction to the theory of numbers}, 5a edici\'o, Oxford University Press, 1979.
%\bibitem{minarc} Helfgott, H. A.: Minor arcs for Goldbach's problem, \newline \texttt{arXiv:1205.5252v4 [math.NT]}, desembre de 2013.
%\bibitem{majarc} Helfgott, H. A.: Major arcs for Goldbach's problem, \newline \texttt{arXiv:1305.2897v4 [math.NT]}, abril de 2014.
%\bibitem{istrue} Helfgott, H. A.: The ternary Goldbach conjecture is true, \newline \texttt{arXiv:1312.7748v2 [math.NT]}, gener de 2014.
%\bibitem{HP} Helfgott, H. A.; Platt, D.: Numerical verification of the ternary Goldbach conjecture up to $8.875 \cdot 10^{30}$, \texttt{arXiv:1305.3062v2 [math.NT]}, abril de 2014.
%\bibitem{KPS} Klimov, N. I.; $\text{Pil}^{\prime} \text{tja}\breve{\imath}$, G. Z.; $\check{\text{S}}\text{eptickaja}$, T. A.: An estimate of the absolute constant in the Goldbach-$\check{\text{S}}\text{nirel}^{\prime} \text{man}$ problem, \textit{Studies in number theory, No. 4}, p\`ag. 35-51, Izdat. Saratov. Univ., Saratov, 1972.
%\bibitem{lw} Liu, M. C.; Wang, T.: On the Vinogradov bound in the three primes Goldbach conjecture, \textit{Acta Arith.}, 105(2):133-175, 2002.
%\bibitem{OSHP} Oliveira e Silva, T.; Herzog, S.; Pardi, S.: Empirical verification of the even Goldbach conjecture and computation of prime gaps up to $4\cdot10^{18}$, \textit{Math. Comp.}, 83:2033-2060, 2014.
%\bibitem{ram} Ramar\'e, O.: On $\check{\text{S}}\text{nirel}^{\prime} \text{man's}$ constant, \textit{Ann. Scuola Norm. Sup. Pisa Cl. Sci.}, 22(4):645-706, 1995.
%\bibitem{riva} Riesel, H.; Vaughan, R. C.: On sums of primes, \textit{Ark. Mat.}, 21(1):46-74, 1983.
%\bibitem{RS} Rosser, J. B.; Schoenfeld, L.: Approximate formulas for some functions of prime numbers, \textit{Illinois J. Math.}, 6:64-94, 1962.
%\bibitem{sch} Schnirelmann, L.: \"Uber additive Eigenschaften von Zahlen, \textit{Math. Ann.}, 107(1):649-690, 1933.
%\bibitem{tao} Tao, T.: Every odd number greater than $1$ is the sum of at most five primes, \textit{Math. Comp.}, 83:997-1038, 2014.
%\bibitem{ari} Travesa, A.: \textit{Aritm\`etica}, Co{\l}ecci\'o UB, No. 25, Barcelona, 1998.
%\bibitem{vau} Vaughan, R. C.: On the estimation of Schnirelman's constant, \textit{J. Reine Angew. Math.}, 290:93-108, 1977.
%\bibitem{vgn} Vaughan, R. C.: \textit{The Hardy-Littlewood method}, Cambridge Tracts in Mathematics, No. 125, 2a edici\'o, Cambridge University Press, 1997.
%\bibitem{vino} Vinogradov, I. M.: Sur le th\'eor\`eme de Waring, \textit{C. R. Acad. Sci. URSS}, 393-400, 1928.
%\bibitem{vin} Vinogradov, I. M.: Representation of an odd number as a sum of three primes, \textit{Dokl. Akad. Nauk. SSSR}, 15:291-294, 1937.

% Cites que s'han de posar si o si
% TensorRingNetsAdaptedDeepMultiTaskLearning (intro mappings de matriu a tensor i tal)
\end{thebibliography}

\end{document}
